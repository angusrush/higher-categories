\documentclass[main.tex]{subfiles}

\begin{document}

\chapter{Introduction}
\label{ch:introduction}

These notes grew out of a course given by Prof.\ Tobias Dyckerhoff at the University of Hamburg. However, much of the presentation and many of the proofs are heavily adapted. In the process I have undoubtedly introduced many mistakes, all of which are entirely my own.

These notes are under very heavy construction at the moment, and the results in them should not be taken as fact. They are a part of my learning process; I write them for this reason.

I write them for another reason as well: I think that there's some pedagogical value in the writings of beginners. The proofs included here are not here because they're the most elegant or the cleanest; instead, they're here because they're the proofs that finally made me, a novice struggling with category theory for the first time, understand the material.

\section{Axiomatic framework}
\label{sec:axiomatic_framework}

These notes are phrased in terms of $\mathbf{ZFCU}$, i.e.\ the Zermelo-Fraenkel axioms together with the following axioms.
\begin{itemize}
  \item $\mathbf{C}$: the axiom of choice.

  \item $\mathbf{U}$: Grothendieck's universe axiom.
\end{itemize}

An introduction to category theory in the language of Grothendieck universes can be found in the corresponding notes on ordinary category theory.\footnote{\url{https://github.com/angusrush/notes-public/blob/master/category_theory/notes.pdf}}

\section{Notational inconsistencies}
\label{sec:notation}

\begin{itemize}
  \item The unit and counit are sometimes called $\epsilon$ and $\eta$ respectively, and sometimes the other way around. I'm going through and fixing this.

  \item For functors $F$, $G\colon \mathcal{C} \to \mathcal{D}$, I sometimes denote the comma category by $(F \downarrow G)$ and sometimes by $\mathcal{C}_{/G}$. I think the first notaion is much better (at least in the context of ordinary categories), but the second seems to be uniquitous in the infinity categorical setting. I haven't decided in which direction to resolve this particular conflict.
\end{itemize}

\end{document}
